\documentclass{article}
\usepackage{amsmath}
\usepackage{mathtools}
\usepackage{amssymb}
\usepackage{algorithm2e}
\usepackage{booktabs}
\usepackage{todonotes}

\newcommand{\eqn}[1]{
	\begin{equation*}
		#1
	\end{equation*}
}

\newcommand{\eqna}[1]{
	\begin{eqnarray*}
		#1
	\end{eqnarray*}
}

\newcommand{\missing}[1]{
	\rule{0.8\textwidth}{.5pt}

	\dots\dots\dots\dots\dots\dots\dots

	#1

	\dots\dots\dots\dots\dots\dots\dots

	\rule{0.8\textwidth}{.5pt}
}


\usepackage{graphicx}
\graphicspath{{images/}}
\DeclareGraphicsExtensions{.pdf,.png,.jpg}

\newcommand{\img}[4]{
	% #1 - source
	% #2 - label
	% #3 - caption
	% #4 width, relative to \textwidth
	\begin{figure}[!t]
		\centering
		\includegraphics[width=#4\textwidth]{#1}
		\caption{#3}
		\label{#2}
	\end{figure}
}

\newcommand{\imgs}[8]{
	\begin{figure}[!tbh]
		\begin{minipage}[b]{#4\textwidth}
			\centering
			\includegraphics[width=\textwidth]{#1}
			\caption{#3}
			\label{#2}
		\end{minipage}
		\hfill
		\begin{minipage}[b]{#8\textwidth}
			\centering
			\includegraphics[width=\textwidth]{#5}
			\caption{#7}
			\label{#6}
		\end{minipage}
	\end{figure}
}

\begin{document}

\section{Adjacency Matrix}

An \emph{adjacency matrix} $A$ is the $|V|\times|V|$ matrix representation of a graph $G=(V,E)$.
As we are only considering simple graphs, i.e., no loops exist, the diagonal alues are all 0, i.e., $A_{i,i} = 0, i \in [1, |V|]$.

$A_{i,j}$ indicates if an edge from $v_i$ to $v_j$ exists, i.e., $A_{i,j} = 1 \equiv (v_i,v_j) \in E$.
In the case of undirected graphs, we simply set $A_{i,j} = A_{j,i}$ which results in an adjacency matrix mirrored at the diagonal, i.e., $A_{i,j} = A_{j,i} = 1 \equiv \{v_i,v_j\} \in E$.

As an example take the following two graphs:

\missing{graph / adjacency matrix example}

The adjacency matrix of a directed graph can be written as a sequence of boolean values of length $|V|\cdot(|V|-1)$.
The adjacency matrix of an undirected graph is mirrored at the diagonal and can therefore be represented as a sequence of length $\frac{|V|\cdot(|V|-1)}{2}$.

\eqn{a = a_1, a_2, \dots a_{\frac{|V|\cdot(|V|-1)}{2}} = A_{1,2}, A_{1,2}, \dots A_{1,|V|}, A_{2,3}, A_{2,4}, \dots A_{|V|-1,|V|}}

As examples, consider the boolean sequence representation for 3-, 4-, and 5-vertex graphs:

\eqna{
	3: && A_{1,2}, A_{1,3}, A_{2,3} = a_1, a_2, a_3 \\
	4: && A_{1,2}, A_{1,3}, A_{1,4}, A_{2,3}, A_{2,4}, A_{3,4} = a_1, a_2, \dots a_6 \\
	5: && A_{1,2}, A_{1,3}, A_{1,4}, A_{1,5}, A_{2,3}, A_{2,4}, A_{2,5}, A_{3,4}, A_{3,5}, A_{4,5} = a_1, a_2, \dots a_{10}
}

Therefore, all possible adjacency matrices for a given graph size $|V|$ can be represented as the set ${\cal A}^{|V|} = \{0,1\}^{\frac{|V|\cdot(|V|-1)}{2}}$.
Each element from this set can also be interpreted as a number $n(a) \in \mathbb{N}$ as follows:

\eqn{n(a) = n(A) = \sum_{i=1}^{\frac{|V|\cdot(|V|-1)}{2}} 2^{i-1} \cdot a_i}

Hence, we can represent adjacency matrices as simple numbers:

\missing{example of adjacency matrix and their key}

\todo{introduce the nummber as ``key'' of an adjacency matrix}
\todo{introduce the induced adjacency matrix of a set of nodes (mentioning the order)}
\todo{introduce a set of all numbers / keys that represent connected adjacency matrices}





\section{Motifs}

We consider an undirected graph to be connected if there exists a path between each pair of vertices.
We denote the set of connected adjacency matrices of $k$-vertex graphs ($k \geq 2$) as ${\cal A}^{|V|}_{con} \subset {\cal A}^{|V|}$.

As motifs of size $k$, also called $k$-vertex motifs or $k$-motifs, we consider the equivalence classes of isomorph connected k-vertex graphs which we denote as ${\cal M}_k$.

Each connected adjacency matrix is assigned to exactly one motif $m \in {\cal M}$.
This assignment can be expressed as a function that maps a connected adjacency matrix to a motif, i.e.,

\eqn{r : {\cal A}^k \rightarrow {\cal M}_k}

\todo{define as mapping from $\mathbb{N}$}

This assignment can be easily computed by enumerating all connected adjacency matrices and determining their equivalence class by performing an isomorphism check with all existing motifs.




\section{Implementation}

For simplicity, we store the function $r$ as integer pairs $(a,b)$ where $a$ is hte key of a connected adjacency matrix and $b$ the equivalence class it belongs to.






\section{Statistics}

\begin{table}[!h]
\begin{tabular}{crrr}
\toprule
nodes	& ams	& motifs	& maxKey \\
\midrule
2	& 1	& 1	& 1 \\
3	& 4	& 2	& 7 \\
4	& 38	& 6	& 63 \\
5	& 728	& 21	& 1023 \\
6	& 26704	& 112	& 32767 \\
7	& 1866256	& 853	& 2097151 \\
\bottomrule
\end{tabular}
\end{table}

\imgs{um-stats}{fig:um-stats}{stats of undirected motifs}{0.4}{um-stats-labeled}{fig:um-stats-labeled}{stats of undirected motifs}{0.4}





\section{k-Neighborhood of an edge}

As the \emph{k-Neighborhood} $N(a,b)$ of an edge $\{a,b\}$, we denote the set of all $(k-2)$-tuples of vertices that are connected to vertices $a$ or $b$.
Thereby, each element of $N(a,b)$ corresponds to a connected subgraph of $G$ that contains $a$ and $b$.

For each induced subgraph $G \cap (\{a,b\} \cup n \in N(a,b))$, we can determine the corresponding motif as follows:

\eqn{r(n(A_{\{a,b\} \cup N(a,b)}))}

in case the edge exists already, we can determine the new motif after the edge removal as follows:

\eqn{r(n(A_{\{a,b\} \cup N(a,b)})-1)}

In case the edge is added, the new state of the motif is computed as follows:

\eqn{r(n(A_{\{a,b\} \cup N(a,b)})+1)}





\section{Algorithm}


\newcommand{\mcount}{{\cal F}}
\newcommand{\mcountof}[1]{{\cal F}(\motif{#1})}
\newcommand{\mcountofat}[2]{{\cal F}_{#2}(\motif{#1})}

\newcommand{\operation}{{\cal O}}
\newcommand{\signature}{{\cal S}}
\newcommand{\signatureofabcd}{\signature(a,b,c,d)}


\begin{algorithm}[!htb]
	\KwData{$G, \{a,b\}, type \in \{add, rm\}$}
	\SetKwComment{Comment}{/* }{ */}
	\Begin{
		\For{$n \in N(a,b)$}{
			\uIf{$type = add$}{
				\Comment*[r]{edge is added}
				${\cal F}(r(n(A_{\{a,b\} \cup n}) + 1))$ += 1 \Comment*[r]{incr new motif}
				\uIf{$n(A_{\{a,b\} \cup n}) \in {\cal N}^k_{con}$}{
					${\cal F}(r(n(A_{\{a,b\} \cup n})))$ -= 1 \Comment*[r]{decr old motif}
				}
			}
      		\uElseIf{$type = rm$}{
      			\Comment*[r]{edge is removed}
				${\cal F}(r(n(A_{\{a,b\} \cup n})))$ -= 1 \Comment*[r]{decr old motif}
       			\uIf{$n(A_{\{a,b\} \cup n})-1 \in {\cal N}^k_{con}$}{
					${\cal F}(r(n(A_{\{a,b\} \cup n})-1))$ += 1 \Comment*[r]{incr new motif}
				}
			}
		}
	}
	\caption{\emph{StreaM$_k$} for maintaining $\mcount$ in dynamic graphs}
	\label{alg:u}
\end{algorithm}
\todo{change set of connected keys to defined version}








\end{document}